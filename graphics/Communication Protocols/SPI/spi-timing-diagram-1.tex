% I2C Timing Diagram for 2 Byte Write.
% By Electro707 (Jamal Bouajjaj)
%
\documentclass[border=10pt]{standalone}
\usepackage{tikz}
\usepackage{tikz-timing}
\usepackage{amsmath}
\usepackage{makecell}

\usetikzlibrary{calc, arrows.meta}

\begin{document}
    \begin{tikztimingtable}[
        timing/lslope=0.1,
        xscale=1.25,
        yscale=1.5,
        semithick,
        grayz/.style={timing/z/.append style={gray}},
        ]
        CLK     & LLL               L N(data1s)16{T}   N(data1e)  L N(data2s)16{T}N(data2e)   L L L\\
        MOSI    & LLL               17D{Data 1 In}  17D{Data 2 In}  L L L\\
        MISO    & ZZN(highzt)LN(cslowEnd)    N(data1mid)17D{Data 1 Out} N(data2mid)17D{Data 2 Out} L L N(cshighEnd)Z\\
        $\overline{\mbox{CS}}$  & N(cslowStart)H N(cstrans)L L 17L 17L N(cshighStart)L N(cs2trans)H H\\
        \extracode
        \makeatletter
        \begin{pgfonlayer}{background}
            \begin{scope}[semitransparent,semithick]
            \fill [red!40] let \p1=(cslowStart), \p2=(cslowEnd) in (\x1,2) rectangle (\x2,-7);
            \fill [blue!40] let \p1=(data1s), \p2=(data1e) in (\x1,2) rectangle (\x2,-7);
            \fill [blue!40] let \p1=(data2s), \p2=(data2e) in (\x1,2) rectangle (\x2,-7);
            \fill [red!40] let \p1=(cshighStart), \p2=(cshighEnd) in (\x1,2) rectangle (\x2,-7);
            \end{scope}
        \end{pgfonlayer}
        { \tiny
        \draw let \p1=($(cstrans)-(1, 0)$) in node[draw, fill=red!30, minimum height=1cm, text width=2cm, anchor=north, align=center] at (\x1, -9) (cstext) {Master pulls the slave's Chip Select line from HIGH to LOW to indicate it wants to communicate with the slave};
        \draw node[draw, fill=red!30, minimum height=1cm, text width=2cm, anchor=north west, align=center] at ($(cstext.north east)+(0.5, 0)$) (highztxt) {Slaves's takes control of the MISO line when selected (compared to it's nominal High-Z state)};
        \draw node[draw, fill=red!30, minimum height=1cm, text width=2cm, anchor=north west, align=center] at ($(highztxt.north east)+(0.5, 0)$) (data1txt) {Master clocks the CLK line 8 times (for a byte) while it's sending data thru the MOSI line at the same time as the slave is potentially sending data back thru the MISO line};
        \draw node[draw, fill=red!30, minimum height=1cm, text width=2cm, anchor=north west, align=center] at ($(data1txt.north east)+(0.5, 0)$) (data2txt) {Again the master clocks the CLK line so that the master can send data out and the slave receive data};
        \draw node[draw, fill=red!30, minimum height=1cm, text width=2cm, anchor=north west, align=center] at ($(data2txt.north east)+(0.5, 0)$) (cs2text) {The master transitions the slave's Chip Select line from LOW to HIGH indicating that the master is done communicating with it};
        \draw node[draw, fill=red!30, minimum height=1cm, text width=2cm, anchor=north west, align=center] at ($(cs2text.north east)+(0.5, 0)$) (highz2txt) {The slave releases the MISO line into a High-Z state};
        }
        
        \draw [black,semithick,-{Triangle}] ($ (cstrans.mid) - (0.1, 0.5) $) parabola [] ($ (cstext.north)$);
        \draw [black,semithick,-{Triangle}] ($ (highzt.mid) - (-0.1, 0.25) $) parabola [bend pos=0] ($ (highztxt.north)$);
        \draw [black,semithick,-{Triangle}] ($ (data1mid.mid) - (-8.9, 0.25) $) parabola [bend pos=0.6] ($ (data1txt.north)$);
        \draw [black,semithick,-{Triangle}] ($ (data2mid.mid) - (-8.9, 0.25) $) parabola [bend pos=0.6] ($ (data2txt.north)$);
        \draw [black,semithick,-{Triangle}] ($ (cs2trans.mid) - (0.1, -0.5) $) parabola [bend pos=0] ($ (cs2text.north)$);
        \draw [black,semithick,-{Triangle}] ($ (cshighEnd.mid) - (0.1, -0.25) $) parabola [bend pos=0] ($ (highz2txt.north)$);
    \end{tikztimingtable}
\end{document}
