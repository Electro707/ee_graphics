% I2C Timing Diagram for 2 Byte Write.
% By Electro707 (Jamal Bouajjaj)
%
\documentclass[border=10pt]{standalone}
\usepackage{tikz}
\usepackage{tikz-timing}
\usepackage{amsmath}
\usepackage{makecell}

\usetikzlibrary{calc}

\begin{document}
    \begin{tikztimingtable}[
        timing/lslope=0.1,
        xscale=1.25,
        yscale=1.5,
        semithick,
        grayz/.style={timing/z/.append style={gray}},
        ]
        SCL     & H 
                N(start-start) HHL N(start-end)
                L L 
                16{T}
                2{T}
                16{T}
                2{T} 
                16{T}
                2{T}
                N(stop-start) LLH  N(stop-end)\\
        SDA     & H 
                HLL              
                L N(addr-start) 14D{7-Bit Address} N(addr-end)
                N(readwrite-start) 2D{R/$\overline{\mbox{W}}$} N(readwrite-end)
                N(ack1-start) 2D{ACK} N(ack1-end)
                N(data1-start) 16D{Data 1} N(data1-end)
                N(ack2-start) 2D{ACK} N(ack2-end)
                N(data2-start) 16D{Data 2} N(data2-end)
                N(ack3-start) 2D{ACK} N(ack3-end)
                L LHH\\
%         Frame   & Z 2D{Start}  \\
        \extracode
        \makeatletter
        \begin{pgfonlayer}{background}
            \begin{scope}[semitransparent,semithick]
            \fill [red!40] let \p1=(start-start), \p2=(start-end) in (\x1,2) rectangle (\x2,-3);
            \fill [orange!40] let \p1=(addr-start), \p2=(addr-end) in (\x1,2) rectangle (\x2,-3);
            \fill [yellow!40] let \p1=(readwrite-start), \p2=(readwrite-end) in (\x1,2) rectangle (\x2,-3);
            \fill [green!40] let \p1=(ack1-start), \p2=(ack1-end) in (\x1,2) rectangle (\x2,-3);
            \fill [blue!40] let \p1=(data1-start), \p2=(data1-end) in (\x1,2) rectangle (\x2,-3);
            \fill [green!40] let \p1=(ack2-start), \p2=(ack2-end) in (\x1,2) rectangle (\x2,-3);
            \fill [blue!40] let \p1=(data2-start), \p2=(data2-end) in (\x1,2) rectangle (\x2,-3);
            \fill [green!40] let \p1=(ack3-start), \p2=(ack3-end) in (\x1,2) rectangle (\x2,-3);
            \fill [red!40] let \p1=(stop-start), \p2=(stop-end) in (\x1,2) rectangle (\x2,-3);
            \end{scope}
        \end{pgfonlayer}
        {
        \tiny
        \draw let \p1=($(start-start)!0.5!(start-end)$) in node[draw, fill=red!30, minimum height=1cm, text width=2cm, anchor=north, align=center] at (\x1, -5) (starttxt) {\textbf{Start Bit}\\Created by pulling SDA low, then SCL low};
        \draw let \p1=($(readwrite-end)+(-0.5,0)$) in node[draw, fill=yellow!30, minimum height=1cm, text width=2cm, anchor=north east, align=center] at (\x1, -5) (rwtxt) {\textbf{Read/Write Bit}\\Set low to read from the slave device, or high to write to it};
        \draw let \p1=($(starttxt.west)!0.5!(rwtxt.east)$) in node[draw, fill=orange!30, minimum height=1cm, text width=2cm, anchor=north, align=center] at (\x1, -5) (starttxt) {\textbf{7-Bit Address}\\The address of the device you want to communicate to};
        \draw let \p1=($(ack1-start)+(0.5,0)$) in node[draw, fill=green!30, minimum height=1cm, text width=2cm, anchor=north west, align=center] at (\x1, -5) (ack1txt) {\textbf{Acknowledgment Bit}\\Set low by the slave which it's address matches with what was sent by the master};
        \draw let \p1=($(ack2-start)$) in node[draw, fill=green!30, minimum height=1cm, text width=4cm, anchor=north west, align=center] at (\x1, -5) (ack2txt) {\textbf{Acknowledgment Bit}\\If the R/$\overline{\mbox{W}}$ bit was set high, the slave writes a LOW if it acknowledges the data. If the R/$\overline{\mbox{W}}$ bit was set low, the master writes a low of it acknowledges the data, or high if it doesn't want to read from the slave anymore};
        \draw let \p1=($(ack1txt.east)!0.5!(ack2txt.west)$) in node[draw, fill=blue!30, minimum height=1cm, text width=2cm, anchor=north, align=center] at (\x1, -5) (starttxt) {\textbf{Data Byte}\\8 Bits of data sent by the master if the R/$\overline{\mbox{W}}$ bit was set high, otherwise it's sent by the slave};
        \draw let \p1=($(stop-start)!0.5!(stop-end)$) in node[draw, fill=red!30, minimum height=1cm, text width=2cm, anchor=north, align=center] at (\x1, -5) (starttxt) {\textbf{Stop Bit}\\Created by pulling SDA high, then SCL high};
        }
%         \node[draw, fill=orange!30, minimum height=1cm, text width=5em, anchor=north west, align=center] at (7, -5) (starttxt) {Address};
    \end{tikztimingtable}
\end{document}
